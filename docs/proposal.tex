



Be sure to motivate your project. Your paper should start by stating the problem. It also should give a use case — essentially, a story about how the approach is used or what to learn from the proposed evaluation. For example, a software developer wants to perform a particular, realistic task, but cannot. Be specific about the problem and about why current techniques do not address it. Your approach solves that problem. An architectural diagram of how the pieces of your system fit together may be a useful summary.

Include your evaluation methodology. The philosopher of science Karl Popper states that a theory is scientific only if it is falsifiable — if some experiment exists that would disprove it. This section should include the research questions (e.g., an experimentally refutable thesis or hypothesis) and how you will answer them. You should also state your evaluation metric, which should be, at least in part, quantifiable. Another way to think of this is: how will you convince a skeptical reader that you succeeded? They aren't going to just take your word that your project was a success.

Clearly state the scientific question or contribution. For example, how is your project an (incremental) advance over previous work? Ideally, your results will be transferable to other domains: they will change the way that others think or act in the future, or provide guidance to programmers or to tool developers or both. Oftentimes, much of your project is not novel. That's OK — engineering is necessary in order to do experiments or to build on previous work — so long as you can point out the part that is novel. A new way of combining previously-known techniques is sufficiently novel. 